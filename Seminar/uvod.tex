\begin{center}
	\textbf{\LARGE{Metoda končnih elementov, ki minimizira kvadrat ostanka aproksimacije (\texttt{LSFEM})}}\\[0.25cm]
	\large{Seminarska naloga pri Naprednih numeričnih metodah}\\[0.7cm]
\end{center}

Numerično reševanje parcialnih diferencialnih enačb (\texttt{PDE}) je zaradi pomanjkanja vsestranskega algoritma še zmeraj bolj umetnost kot ustaljena znanost \cite{JiangB-LSFEM}. Pri zapletenih problemih hitro prispemo do vznožja gore matematične teorije, ki je ni moč zaobiti. Zaradi množice različnih pristopov reševanja ter raztresene in neprijazno napisane literature, lahko le ugibamo, kako visoko se bomo na poti do prelaza morali povzpeti. Zapletenim problemom prostorske dinamike v:
\begin{center}
	\begin{tabular}[h]{lll}
		\tabitem dinamiki tekočin,\hspace{1cm}	&	\tabitem termodinamiki,\hspace{2.5cm}	&	\tabitem elektrodinamiki,\\
		\tabitem kvantni teoriji,	&	\tabitem splošni teoriji relativnosti,&	\\
	\end{tabular}
\end{center}
kjer naletimo na \texttt{PDE}, se tako tudi v višjem izobraževanju najraje izognemo. Metoda končnih elementov (\texttt{FEM}), ki minimizira kvadrat ostanka aproksimacije (\texttt{LSFEM}: Least Squares \texttt{FEM}), obeta razvoj vsestranskega algoritma za reševanje \texttt{PDE} in s tem približanje omenjenih problemov širšemu krogu raziskovalcev.