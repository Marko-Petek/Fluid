\section{Diskretizacija}

Od te točke dalje nadaljujemo z diskretizacijo problema, to je, pretvorbo na sistem $N$ algebrajskih enačb. Ta korak je enak pri vseh različicah \texttt{FEM}. Funkcije na domeni $\Omega$ imajo neskončno štveilo prostostnih stopenj. Problem pripravimo za numerično reševanje tako, da funkcijam  omejimo število prostostnih stopenj. Rešitev npr. zapišemo kot sestavljanko vozliščnih funkcij. Imamo $N$ vozlišč:
\begin{equation}
    u_i(\mathbf{x}) = \sum_{a = 1}^N \Phi^{a0} u^{a0}_i
\end{equation}

Potem omejimo Diskretizacija problema 

Galerkin, Najmanših kvadratov \cite{JiangB-LSFEM}
Basic lemma of variational principles: Temeljni lema variacijskih načel.