\section{Diskretizacija problema}

Z natančno analitično izpeljavo se prebijemo do izjave \eqref{eq:LsfemVariationalStatement}, od tod dalje pa moramo iskanje funkcije $\mathbf{u(x)}$ z neskončno prostostnimi stopnjami poenostaviti v iskanje funkcije s končnim številom prostostnih stopenj $N$. Poljubno funkcijo $f(\mathbf{x})$ na domeni $\Omega$ zapišemo kot sestavljanko N vozliščnih funkcij $\Phi_i$:
\begin{equation}
    f(\mathbf{x}) = \sum_{i = 1}^N f_i  \mkern1mu \Phi_i
\end{equation}

nadaljujemo z diskretizacijo problema, to je, pretvorbo na sistem $N$ algebrajskih enačb. Ta korak je enak pri vseh različicah \texttt{FEM}. Funkcije na domeni $\Omega$ imajo neskončno štveilo prostostnih stopenj. 
\begin{equation}
    u_i(\mathbf{x}) = \sum_{a = 1}^N \Phi^{a0} u^{a0}_i
\end{equation}

Potem omejimo Diskretizacija problema 

Galerkin, Najmanših kvadratov \cite{JiangB-LSFEM}
Basic lemma of variational principles: Temeljni lema variacijskih načel.